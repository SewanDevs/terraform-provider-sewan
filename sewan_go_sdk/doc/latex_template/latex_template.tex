%
% Compilation of the examples from the TikZ-UML manual, v. 1.0b (2013-03-01)
% http://www.ensta-paristech.fr/~kielbasi/tikzuml/index.php?lang=en
%
\documentclass[a4paper,12pt]{article}

\usepackage[T1]{fontenc}
\usepackage[utf8]{inputenc}
\usepackage[english]{babel}
\usepackage{fullpage}

\usepackage{tikz-uml}

\sloppy
\hyphenpenalty 10000000

\title{Package Example: TikZ UML Diagrams}
\author{Nicolas Kielbasiewicz}

\begin{document}

\maketitle

\section{Class Diagram}

\begin{center}
\begin{tikzpicture}
\begin{umlpackage}{p}
\begin{umlpackage}{sp1}
\umlclass[template=T]{A}{
  n : uint \\ t : float
}{}
\umlclass[y=-3]{B}{
  d : double
}{
  \umlvirt{setB(b : B) : void} \\ getB() : B}
\end{umlpackage}
\begin{umlpackage}[x=10,y=-6]{sp2}
\umlinterface{C}{
  n : uint \\ s : string
}{}
\end{umlpackage}
\umlclass[x=2,y=-10]{D}{
  n : uint
  }{}
\end{umlpackage}

\umlassoc[geometry=-|-, arg1=tata, mult1=*, pos1=0.3, arg2=toto, mult2=1, pos2=2.9, align2=left]{C}{B}
\umlunicompo[geometry=-|, arg=titi, mult=*, pos=1.7, stereo=vector]{D}{C}
\umlimport[geometry=|-, anchors=90 and 50, name=import]{sp2}{sp1}
\umlaggreg[arg=tutu, mult=1, pos=0.8, angle1=30, angle2=60, loopsize=2cm]{D}{D}
\umlinherit[geometry=-|]{D}{B}
\umlnote[x=2.5,y=-6, width=3cm]{B}{Je suis une note qui concerne la classe B}
\umlnote[x=7.5,y=-2]{import-2}{Je suis une note qui concerne la relation d'import}
\end{tikzpicture}
\end{center}

\section{Sequence Diagram}

\begin{center}
\begin{tikzpicture}
\begin{umlseqdiag}
\umlactor[class=A]{a}
\umldatabase[class=B, fill=blue!20]{b}
\umlmulti[class=C]{c}
\umlobject[class=D]{d}
\begin{umlcall}[op=opa(), type=synchron, return=0]{a}{b}
\begin{umlfragment}
\begin{umlcall}[op=opb(), type=synchron, return=1]{b}{c}
\begin{umlfragment}[type=alt, label=condition, inner xsep=8, fill=green!10]
\begin{umlcall}[op=opc(), type=asynchron, fill=red!10]{c}{d}
\end{umlcall}
\begin{umlcall}[type=return]{c}{b}
\end{umlcall}
\umlfpart[default]
\begin{umlcall}[op=opd(), type=synchron, return=3]{c}{d}
\end{umlcall}
\end{umlfragment}
\end{umlcall}
\end{umlfragment}
\begin{umlfragment}
\begin{umlcallself}[op=ope(), type=synchron, return=4]{b}
\begin{umlfragment}[type=assert]
\begin{umlcall}[op=opf(), type=synchron, return=5]{b}{c}
\end{umlcall}
\end{umlfragment}
\end{umlcallself}
\end{umlfragment}
\end{umlcall}
\umlcreatecall[class=E, x=8]{a}{e}
\begin{umlfragment}
\begin{umlcall}[op=opg(), name=test, type=synchron, return=6, dt=7, fill=red!10]{a}{e}
\umlcreatecall[class=F, stereo=boundary, x=12]{e}{f}
\end{umlcall}
\begin{umlcall}[op=oph(), type=synchron, return=7]{a}{e}
\end{umlcall}
\end{umlfragment}
\end{umlseqdiag}
\end{tikzpicture}
\end{center}

\section{State Diagram}

\begin{center}
\begin{tikzpicture}
\begin{umlstate}[name=Amain]{Etat global de l'objet A}
\begin{umlstate}[name=Bgraph, fill=red!20]{graphe B}
\umlstateinitial[name=Binit]
\umlbasicstate[y=-4, name=test1, fill=white]{test1}
\umltrans{Binit}{test1}
\umltrans[recursive=20|60|2.5cm, recursive direction=right to top, arg={op1}, pos=1.5]{test1}{test1}
\umltrans[recursive=160|120|2.5cm, recursive direction=left to top, arg={op2}, pos=1.5]{test1}{test1}
\umltrans[recursive=-160|-120|2.5cm, recursive direction=left to bottom, arg={op3}, pos=1.5]{test1}{test1}
\umltrans[recursive=-20|-60|2.5cm, recursive direction=right to bottom, arg={op4}, pos=1.5]{test1}{test1}
\umlbasicstate[y=-8, name=test2, fill=white]{test2}
\umltrans[recursive=-160|-120|2.5cm, recursive direction=left to bottom, arg={op5}, pos=1.5]{test2}{test2}
\umltrans{test1}{test2}
\umlstatefinal[x=3, y=-7.75, name=Bfinal]
\umltrans{test2}{Bfinal}
\end{umlstate}
\umlstateinitial[x=6, y=1, name=Ainit]
\umlVHtrans[anchor2=40]{Ainit}{Bgraph}
\umlstatefinal[x=6, y=-3.5, name=Afinal]
\umlHVtrans[anchor1=30]{Bgraph}{Afinal}
\umlbasicstate[x=6, y=-6, name=visu, fill=green!20]{Visualisation}
\umlHVtrans{Bfinal}{visu}
\umltrans{visu}{Afinal}
\umltrans[recursive=-20|-60|2.5cm, recursive direction=right to bottom, arg=a, pos=1.5]{visu}{visu}
\end{umlstate}
\end{tikzpicture}
\end{center}

\section{Component Diagram}

\begin{center}
\begin{tikzpicture}
\begin{umlcomponent}{A}
\umlbasiccomponent{B}
\umlbasiccomponent[y=-2]{C}

\umlrequiredinterface[interface=C-interface]{C}
\umlprovidedinterface[interface=B-interface, with port, distance=3cm, padding=2.5cm]{B}
\end{umlcomponent}
\umlbasiccomponent[x=-10,y=1]{D}
\umlbasiccomponent[x=3,y=-7.5]{E}
\umlbasiccomponent[x=-2, y=-9]{F}
\umlbasiccomponent[x=-7,y=-8]{G}
\umlbasiccomponent[x=-7,y=-11]{H}

\umlassemblyconnector[interface=DA, with port, name=toto]{D}{A}
\umldelegateconnector{A-west-port}{B-west-interface}
\umlVHVassemblyconnector[interface=AE, with port]{A}{E}
\umlHVHassemblyconnector[interface=EF, with port, first arm]{E}{F}
\umlHVHassemblyconnector[interface=GHF, with port, arm2=-2cm, last arm]{G}{F}
\umlHVHassemblyconnector[with port, arm2=-2cm, last arm]{H}{F}

\umlnote[x=-4, y=4, width=3.4cm]{B-west-interface}{I am the node named B-west-interface}
\umlnote[x=2, y=4, width=3.4cm]{C-east-interface}{I am the node named C-east-interface}
\umlnote[x=-8.5, y=-2, width=3.4cm]{toto-interface}{I am the node named toto-interface}
\umlnote[x=-5.5, y=-4.5, width=3.4cm]{A-south-port}{I am the node named A-south-port}
\umlnote[x=-1, y=-6, width=3.4cm]{AE-interface}{I am the node named AE-interface}
\umlnote[x=2, y=-11, width=3.4cm]{F-east-port}{I am the node named F-east-port}
\end{tikzpicture}
\end{center}

\end{document}